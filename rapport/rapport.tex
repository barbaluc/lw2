\documentclass{univ-projet}


%Import des packages utilisés pour le document
\usepackage[utf8x]{inputenc}
\usepackage[francais]{babel}
\usepackage[T1]{fontenc}
\usepackage{verbatim}
%\usepackage{array}
%\usepackage{hyperref}
%\usepackage{tabularx, longtable}
%\usepackage[table]{xcolor}
%\usepackage{fancyhdr}
%\usepackage{lastpage}

\usepackage{amsmath,amsfonts,amssymb,amsthm,epsfig,epstopdf}



\definecolor{gris}{rgb}{0.95, 0.95, 0.95}

%Redéfinition des marges
%\addtolength{\hoffset}{-2cm}
%\addtolength{\textwidth}{4cm}
\addtolength{\topmargin}{-1cm}
\addtolength{\textheight}{1cm}
\addtolength{\headsep}{0.8cm} 
\addtolength{\footskip}{-0.2cm}


\theoremstyle{definition}
\newtheorem{defn}{Definition}[section]

%Import page de garde et structures pour la gestion de projet
%\usepackage{structures}

%Variables
\logo{logo_univ.png}
\title{Rapport de projet}
\author{Lucas \bsc{Barbay}}
\projet{Langages Web 2}
\projdesc{Client-serveur REST}
\filiere{M1 GIL}
\signataire{}
\date{\today}


% -- Début du document -- %
\begin{document} 

%Page de garde
\maketitle
\newpage
%La table des matières
\tableofcontents
\newpage

\section{L'application}

Vous la trouverais à l'url suivante : \url{http://lw2-barbaluc.rhcloud.com/}. L'application est hébergée sur un serveur openshift par redhat. J'ai décidé d'utiliser openshift car nous l'avions déjà utilisé en architecture distribuée.

Pour retrouver les sources de mon application, le dépôt github : \url{https://github.com/barbaluc/lw2}

L'application contient les trois méthodes demandées : 

\begin{itemize}
\item /resume qui liste tous les CV présent dans la base de données
\item /resume/id qui affiche le CV contenant l'id en paramètre
\item /resume qui créé un nouveau CV et l'ajoute dans la liste par le biais de la méthode POST
\end{itemize}

La persistence est disponible jusqu'à l'extinction du serveur openshift.


\section{Le client}

Le client est une interface en java permettant la création de CV par le biais d'un formulaire et l'envoie de ce CV sur l'application grâce à la méthode POST.

Les sources de mon interface sont disponible à l'adresse suivante : \url{https://github.com/barbaluc/lw2client}

\end{document}